\makeatother

\newcommand{\varn}[1]{\textsl{#1}}

\newcommand{\crefApx}[1]{\cref{#1} (Appendix)}
\newcommand{\CrefApx}[1]{\Cref{#1} (Appendix)}

\newcommand{\hc}[1]{#1}
\newcommand{\hcm}[1]{\hc{\ensuremath{#1}}\xspace}
\newcommand{\hcs}[1]{\hc{#1}\xspace}
\newcommand{\formatIdentifier}[1]{\hcm{\mathsf{#1}}}

\renewcommand{\epsilon}{\varepsilon}
\newcommand{\remove}[1]{}

\newcommand{\lbbfs}{{\cal D}_{\mathrm{BFS}(v)}}
\newcommand{\size}{\mathrm{size}}
\newcommand{\dpth}{\mathrm{depth}}
\newcommand{\maxdpth}{\overline{\dpth}}


\DeclareMathOperator{\polylog}{poly\,log}
\DeclareMathOperator{\sort}{sort}
\DeclareMathOperator{\scan}{scan}
\DeclareMathOperator{\cdm}{cdm}
\DeclareMathOperator{\acosh}{acosh}
\DeclareMathOperator{\asinh}{asinh}
\let\arccos\relax
\DeclareMathOperator{\arccos}{acos}
\DeclareMathOperator{\odeg}{outDeg}
\DeclareMathOperator{\avg}{avg}

\newcommand{\qq}[1]{\glqq #1\grqq\xspace}

\newcommand{\symOt}{\ensuremath{\tilde{\mathcal{O}}}}
\newcommand{\symOh}{\ensuremath{\mathcal{O}}}
\newcommand{\symoh}{\ensuremath{{o}}}
\newcommand{\symTh}{\ensuremath{\Theta}}
\newcommand{\symOm}{\ensuremath{\Omega}}
\newcommand{\symom}{\ensuremath{\omega}}

\newcommand{\Ot}[1]{\ensuremath{\symOt\!\left(#1\right)}}
\newcommand{\Oh}[1]{\ensuremath{\symOh\!\left(#1\right)}}
\newcommand{\oh}[1]{\ensuremath{\symoh\!\left(#1\right)}}
\newcommand{\Th}[1]{\ensuremath{\symTh\!\left(#1\right)}}
\newcommand{\Om}[1]{\ensuremath{\symOm\!\left(#1\right)}}
\newcommand{\om}[1]{\ensuremath{\symom\!\left(#1\right)}}

\newcommand{\OhSq}[1]{\ensuremath{\symOh\!\left[#1\right]}}

\newcommand{\I}{i}
\newcommand{\setd}[2]{\ensuremath{\left\{#1 \, \middle|\, #2\right\}}}
\newcommand{\acos}{\arccos}


\newcommand{\intd}{\text d}

\newcommand{\e}{\ensuremath{e}}
\newcommand{\errorterm}[1]{#1}%{\textcolor{blue}{ #1 }}
\newcommand{\req}[2]{\ensuremath{\texttt{req}_{#1}(#2)}}

% abbr
% machine models
\newcommand{\nproc}{\ensuremath{P}}

% graph
%\newcommand{\opstyle}[1]{\mathrm{\textcolor{blue}{#1}}}
\newcommand{\opstyle}[1]{\mathrm{#1}}
\newcommand{\degree}{\ensuremath{\opstyle{deg}}}
\newcommand{\degin} {\ensuremath{\opstyle{deg^{in}}}}
\newcommand{\degout}{\ensuremath{\opstyle{deg^{out}}}}
\newcommand{\maxdeg}{\ensuremath{\opstyle{maxdeg}}}
\newcommand{\diam}{\ensuremath{\opstyle{diam}}}
\newcommand{\maxdiam}{\overline{\diam}}
\newcommand{\dist}{\ensuremath{\opstyle{dist}}}
\newcommand{\adj}{\ensuremath{\opstyle{N}}}
\newcommand{\adjin}{\ensuremath{\opstyle{N^{in}}}}
\newcommand{\adjout}{\ensuremath{\opstyle{N^{out}}}}
\newcommand{\degseq}{\ensuremath{\mathcal{D}}}
\newcommand{\cc}{\ensuremath{\opstyle{cc}}}
\newcommand{\density}{\ensuremath{\opstyle{dens}}}

% graph classes
\newcommand{\gGn} {\texorpdfstring{\ensuremath{\mathbb G(n)}}{G(n)}\xspace} % graphs on n vertices
\newcommand{\gGnm}{\texorpdfstring{\ensuremath{\mathbb G(n,m)}}{G(n,m)}\xspace} % graphs on n vertices with m edges
\newcommand{\gGnr}{\texorpdfstring{\ensuremath{\mathbb G^{(r)}(n)}}{Gr(n)}\xspace} % r-regular graphs on n vertices
\newcommand{\Gn}  {\texorpdfstring{\ensuremath{\mathcal{G}(n)}}{G(n)}\xspace} % uniform distribution on \gGn
\newcommand{\Gnm} {\texorpdfstring{\ensuremath{\mathcal{G}(n,m)}}{G(n,m)}\xspace} % uniform distribution on \gGnm
\newcommand{\Gnp} {\texorpdfstring{\ensuremath{\mathcal{G}(n,p)}}{G(n,p)}\xspace}
\newcommand{\Gnr} {\texorpdfstring{\ensuremath{\mathcal{G}^{(r)}(n)}}{Gr(n)}\xspace} % uniform distribution on \gGnr
\newcommand{\gd} {\texorpdfstring{\ensuremath{\mathbb{G}(\degseq)}}{G(D)}\xspace} % uniform distribution on \gGnr
\newcommand{\Gd} {\texorpdfstring{\ensuremath{\mathcal{G}(\degseq)}}{G(D)}\xspace} % uniform distribution on \gGnr

\newcommand{\CMd} {\texorpdfstring{\ensuremath{\mathcal{CM}(\degseq)}}{CM Tr011(D)}\xspace} % uniform distribution on \gGnr
\newcommand{\CMnd} {\texorpdfstring{\ensuremath{\mathcal{CM}(n, \degseq)}}{CM(n, D)}\xspace} % uniform distribution on \gGnr


% common sets
\newcommand{\sN}{\ensuremath{\mathbb{N}}}
\newcommand{\sNpos}{\ensuremath{\mathbb{N}_{>0}}}
\newcommand{\sZ}{\ensuremath{\mathbb{Z}}}
\newcommand{\sR}{\ensuremath{\mathbb{R}}}
\newcommand{\NP}{\ensuremath{\mathcal{NP}}}

% prop
\newcommand{\symprob}{\ensuremath{\mathbb{P}}}
\newcommand{\symexpv}{\ensuremath{\mathbb{E}}}
\newcommand{\prob}[1]{\ensuremath{\symprob\!\left[#1\right]}}
\newcommand{\expv}[1]{\ensuremath{\symexpv\!\left[#1\right]}}
\newcommand{\varv}[1]{\ensuremath{\operatorname{Var}\!\left[#1\right]}}
\newcommand{\pld}[3]{\ensuremath{\textsc{Pld}\,([#1, #2), #3)}}
\newcommand{\binomd}[2]{\ensuremath{\text{\textsc{BinD}}\left[#1, #2\right]}}
\newcommand{\geom}[1]{\ensuremath{\opstyle{Geom}(#1)}}
\newcommand{\whp}{whp.\xspace}
\newcommand{\follows}{\sim}


\newcommand{\Def}{\ensuremath{:=}}
\newcommand{\seq}[3]{\ensuremath{[\,#1\,]_{#2}^{#3}}}
\newcommand{\rank}{\ensuremath{\operatorname{rank}}}
\newcommand{\var}{\ensuremath{\operatorname{Var}}}

\newcommand{\mean}[1]{\ensuremath{\left\langle #1 \right\rangle}}
\newcommand{\avgdeg}{\ensuremath{\avg{\degree}}}
\newcommand{\avgdist}{\ensuremath{\avg{\dist}}}

\newcommand{\set}[1]{\ensuremath{\left\{ #1\right\}}}
\newcommand{\twoset}[2]{\ensuremath{\left\{ #1 \ \ \middle|\ \ #2 \right\}}}
\newcommand{\floor}[1]{\ensuremath{\left\lfloor #1\right\rfloor}}
\newcommand{\card}[1]{\ensuremath{\lvert #1\rvert}}

\newcommand{\showcomment}[1]{#1}
\renewcommand{\showcomment}[1]{} % Comment in to disable comments
\newcommand{\citeneed}{\showcomment{\ensuremath{^\text{\color{red}{[citation needed]}}}}\xspace}

\def\ld{\ensuremath{\log_2}}




\usepackage{todonotes}
\newcommand{\TODO}[1]{\todo[inline]{{\footnotesize #1}}}

\newcommand*{\diff}{\mathop{}\!\mathrm{d}}


\newcommand{\invar}[1]{(\texttt{I#1})}

\newcommand{\degs}{\ensuremath{\mathcal D}}
\newcommand{\defrel}{:=}

\newenvironment{lemmarep}[1]{\noindent\textbf{Lemma~\ref{#1}. (repeated)}\itshape}{}
\newcommand{\mult}[1]{\#(#1)}
\newcommand{\moment}[1]{\langle #1 \rangle}

\newcommand{\sequence}[3]{\ensuremath{[\,#1\,]_{#2}^{#3}}}

\DeclareMathOperator{\swap}{\mathtt{swap}}
\DeclareMathOperator{\fst}{fst}
\DeclareMathOperator{\snd}{snd}

\newcommand{\formatImpl}[1]{\textsc{#1}}

\newcommand{\dhrg}{\ensuremath{d_{\mathrm{HRG}}}}
\newcommand{\dgirg}{\ensuremath{d_{\mathrm{GIRG}}}}
\newcommand{\Dgirg}{\ensuremath{D_{\mathrm{GIRG}}}}
\DeclareMathOperator{\level}{level}

\newcommand{\BigO}{\mathcal{O}}
\newcommand{\erdos}{Erd{\H o}s-R{\'e}nyi}

\newcommand{\punkt}{\enspace .}
\newcommand{\komma}{\enspace ,}
\newcommand{\expect}{{\mathbf{E}}}
% asymptotische Notation
\newcommand{\whpO}[1]{\tilde{\mathrm{O}}\left( #1\right)}
\newcommand{\Oschlange}{$\tilde{\mathrm{O}}$}

\newcommand{\Ohh}[1]{\mathcal{O}\!\left( #1\right)}
\newcommand{\Ohsmall}[1]{\mathcal{O}(#1)}
\newcommand{\Thsmall}[1]{\Theta( #1)}
\newcommand{\Omsmall}[1]{\Omega( #1)}
\newcommand{\Oleq}{\preceq}

\newcommand{\CC}{C\raisebox{.08ex}{\hbox{\tt ++}}}
\newcommand{\Cminus}{C\raisebox{.08ex}{\hbox{\tt ++}}\raisebox{.25ex}{-}}
\newcommand{\GG}{g\raisebox{.08ex}{\hbox{\tt ++}}}
\newcommand{\Gminus} {g\raisebox{.08ex}{\hbox{\tt ++}}\raisebox{.25ex}{-}}




\newcommand{\A}{\mathcal A}
\newcommand{\Nei}[1]{\ensuremath{\A_{#1}}}
\newcommand{\AG}{\mathscr{A}_G}
\newcommand{\Suc}{\ensuremath{\mathscr{S}}}
\renewcommand{\O}{\mathcal O} % big O

\newcommand{\hide}[1]{}

\newcommand\myalgorithmCommSty[1]{\footnotesize\ttfamily\textcolor{olive}{#1}}


\let\epsilon\varepsilon


\def\Wlog      {\hcs{W.l.o.g.}\ }
\def\wrt       {\hcs{w.\,r.\,t.}}
\def\etal      {\hcs{et~al.}}
\def\eg        {\hcs{e.g.,}}
\def\cf        {\hcs{cf.}\ }
\def\ie        {\hcs{i.e.,}}
\def\citeneeded{\textsuperscript{\textcolor{red}{[cite needed]}}}

% Deutsche Abkürzungen
\def\zB   	{\hcs{z.\,B.}}
\def\ZB     {\hcs{Z.\,B.}}
\def\dh		{\hcs{d.\,h.}}
\def\Dh 	{\hcs{D.\,h.}}
\def\idR 	{\hcs{i.\,d.\,R.}}
\def\IdR 	{\hcs{I.\,d.\,R.}}
\def\oBdA 	{\hcs{o.\,B.\,d.\,A.}}
\def\OBdA 	{\hcs{O.\,B.\,d.\,A.}}
\def\oA 	{\hcs{o.\,Ä.}}
\def\sd 	{\hcs{s.\,d.}}

\newcommand{\Ex}[1]{\hcm{\mathbb E\left[#1\right]}}
\newcommand{\Prob}[1]{\hcm{\mathbb P\left[#1\right]}}


\renewcommand{\set}[1]{\hcm{\left\{#1\right\}}}
\newcommand{\problem}[1]{\textsc{#1}}

\DeclareMathOperator{\poly}{poly}


\newcommand*{\LDAUOmicron}[1]{\ensuremath{\operatorname{O}\left(#1\right)}}
\newcommand*{\LDAUomicron}[1]{\ensuremath{\operatorname{o}\left(#1\right)}}
\newcommand*{\LDAUOmega}[1]{\ensuremath{\operatorname{\Omega}\left(#1\right)}}
\newcommand*{\LDAUomega}[1]{\ensuremath{\operatorname{\omega}\left(#1\right)}}
\newcommand*{\LDAUTheta}[1]{\ensuremath{\operatorname{\Theta}\left(#1\right)}}

\newcommand*{\LdauOmicron}[1]{\ensuremath{\operatorname{O}\bigl(#1\bigr)}}
\newcommand*{\Ldauomicron}[1]{\ensuremath{\operatorname{o}\bigl(#1\bigr)}}
\newcommand*{\LdauOmega}[1]{\ensuremath{\operatorname{\Omega}\bigl(#1\bigr)}}
\newcommand*{\Ldauomega}[1]{\ensuremath{\operatorname{\omega}\bigl(#1\bigr)}}
\newcommand*{\LdauTheta}[1]{\ensuremath{\operatorname{\Theta}\bigl(#1\bigr)}}

\newcommand*{\ldauOmicron}[1]{\ensuremath{\operatorname{O}(#1)}}
\newcommand*{\ldauomicron}[1]{\ensuremath{\operatorname{o}(#1)}}
\newcommand*{\ldauOmega}[1]{\ensuremath{\operatorname{\Omega}(#1)}}
\newcommand*{\ldauomega}[1]{\ensuremath{\operatorname{\omega}(#1)}}
\newcommand*{\ldauTheta}[1]{\ensuremath{\operatorname{\Theta}(#1)}}

\newcommand*{\SOFTOmicron}[1]{\ensuremath{\operatorname{\tilde{O}}\left(#1\right)}}
\newcommand*{\SOFTomicron}[1]{\ensuremath{\operatorname{\tilde{o}}\left(#1\right)}}
\newcommand*{\SOFTOmega}[1]{\ensuremath{\operatorname{\tilde{\Omega}}\left(#1\right)}}
\newcommand*{\SOFTomega}[1]{\ensuremath{\operatorname{\tilde{\omega}}\left(#1\right)}}
\newcommand*{\SOFTTheta}[1]{\ensuremath{\operatorname{\tilde{\Theta}}\left(#1\right)}}

\newcommand*{\SoftOmicron}[1]{\ensuremath{\operatorname{\tilde{O}}\bigl(#1\bigr)}}
\newcommand*{\Softomicron}[1]{\ensuremath{\operatorname{\tilde{o}}\bigl(#1\bigr)}}
\newcommand*{\SoftOmega}[1]{\ensuremath{\operatorname{\tilde{\Omega}}\bigl(#1\bigr)}}
\newcommand*{\Softomega}[1]{\ensuremath{\operatorname{\tilde{\omega}}\bigl(#1\bigr)}}
\newcommand*{\SoftTheta}[1]{\ensuremath{\operatorname{\tilde{\Theta}}\bigl(#1\bigr)}}

\newcommand*{\softOmicron}[1]{\ensuremath{\operatorname{\tilde{O}}(#1)}}
\newcommand*{\softomicron}[1]{\ensuremath{\operatorname{\tilde{o}}(#1)}}
\newcommand*{\softOmega}[1]{\ensuremath{\operatorname{\tilde{\Omega}}(#1)}}
\newcommand*{\softomega}[1]{\ensuremath{\operatorname{\tilde{\omega}}(#1)}}
\newcommand*{\softTheta}[1]{\ensuremath{\operatorname{\tilde{\Theta}}(#1)}}

\DeclarePairedDelimiter{\ceil}{\lceil}{\rceil}

\def\?#1{}

%\usepackage{enumitem}
\newenvironment{itemize*}%
{\begin{itemize}%[leftmargin=1.5em]%
		%		\setlength{\itemsep}{0pt}%
		%		\setlength{\parskip}{0pt}%
		%		\setlength{\topsep}{1pt}%
		%		\setlength{\partopsep}{1pt}%
		}{\end{itemize}}

\newenvironment{enumerate*}%
{\begin{enumerate}%[leftmargin=2.5em]%
		%		\setlength{\itemsep}{0pt}%
		%		\setlength{\parskip}{0pt}%
		%		\setlength{\topsep}{1pt}%
		%		\setlength{\partopsep}{1pt}%
		}{\end{enumerate}}


\newenvironment{myalgorithm}[1][htb]
{
	\let\oldnl\nl% Store \nl in \oldnl
	\newcommand{\nonl}{\renewcommand{\nl}{\let\nl\oldnl}}% Remove line number for one line

	\SetCommentSty{myalgorithmCommSty}

	\DontPrintSemicolon
	\SetAlgoLined\SetAlgoNoEnd
	\SetArgSty{normalfont}
	\SetKwInOut{Input}{Input}
	\SetKwInOut{Output}{Output}
	\SetKw{Continue}{continue}
	\SetKwFor{ParForEach}{par foreach}{}{}
	\SetKw{Sequentially}{sequentially}
	\renewcommand{\algorithmcfname}{Algorithm}% Update algorithm name
	\begin{algorithm}[#1]\small%
		}{\end{algorithm}}

\def\mainchapters{\Cref{cpt:pa,cpt:emlfr,cpt:curveball,cpt:hypergen,cpt:commfree,cpt:girgs}}


