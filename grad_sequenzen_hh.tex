\section{Charakterisierungen graphischer Sequenzen}
Wir können graphische Gradsequenzen anhand mehrerer Eigenschaften charakterisieren, wobei das Erd\H{o}s-Gallai Theorem sowie das Havel-Hakimi Theorem wohl die bekanntesten Ansätze darstellen.
Das Erd\H{o}s-Gallai Theorem ist nicht-konstruktiv und kommt daher in der Praxis besonders zum Testen auf graphische Sequenzen zum Einsatz.
Aus dem Havel-Hakimi Theorem, hingegen, können wir einen Graphgenerator konstruieren.
Die Intuition von beiden Ansätzen ist aber ähnlich.

\subsection{Erd\H{o}s-Gallai Theorem}
\begin{theorem}
    Sei $\degseq = (d_1, \ldots, d_n)$ eine nicht-wachsende Gradsequenz, d.h.
    \begin{align}
        n > d_1 \ge d_2 \ge \ldots \ge d_n \ge 1.
    \end{align}
    Dann ist~$\degseq$ genau dann graphisch, wenn folgende beide Bedingungen erfüllt sind:
    \begin{enumerate}
        \item Die Summe der Grade $\sum_{i=1}^n d_i$ ist gerade (durch 2 teilbar)
        \item Für alle $1 \le k \le n$ gilt $\sum_{i=1}^k d_i \le k(k-1) + \sum_{i=k+1}^n \min(k, d_i)$\hfill\qedhere
    \end{enumerate}
\end{theorem}

\begin{exercise}
    Beschreibe einen möglichst effizienten Algorithmus, der das Erd\H{o}s-Gallai Theorem nutzt, um zu entscheiden, ob eine Gradsequenz~$\degseq$ graphisch ist.
    Analysiere dessen Laufzeit.
\end{exercise}

Im Folgenden diskutieren wir kurz, weshalb das Erd\H{o}s-Gallai Theorem eine notwendige Bedingung formuliert (d.h. jede graphisch Sequenz erfüllt das Theorem).
Die Gegenrichtung (d.h. jede erfüllende Gradsequenz ist graphisch) ist deutlich komplexer und wird hier nicht behandelt.

Die Anforderung, dass die Gradsumme gerade sein muss, folgt trivial aus dem Handshaking Lemma:
jede Kante hat zwei Endpunkte und erhöht somit die Gradsumme um zwei
--- eine Ausnahme wäre es lediglich, wenn wir Eigenschleifen nur als Grad~$1$ zählen würden; das ist aber egal, da einfache Graphen sowieso keine Schleifen besitzen.

Für die zweite Bedingung konzentrieren wir uns zunächst auf die Ungleichung für $k = n$.
Diese lautet
\begin{align}
    \sum_{i=1}^n d_i \le n(n-1) + 0
\end{align}

Dies ist trivial richtig, da nur der vollständige Graph $K_n$ mit $\degseq = (n{-}1, \ldots, n{-}1)$ die Ungleichung mit Gleichheit erfüllt.
Für jeden anderen Graphen ist die linke Seite der Ungleichung echt kleiner.

Für $k < n$, beschreibt also der $k(k-1)$ Term der rechten Seite die Grade, die durch Kanten innerhalb der ersten $k$ Knoten erklärt werden können.
Falls ein $k$ existiert mit $\sum_{i=1}^k (d_i) - k(k-1) = \Delta > 0$, müssen $\Delta$ Kanten die ersten $k$ Knoten mit den verbleibenden $n - k$ verbinden.
Es muss also $\sum_{i=k+1}^k d_i \ge \Delta$ gelten; tatsächlich muss die Grenze aber noch schärfer gewählt werden.
Da wir keine Mehrfachkanten erlauben, darf jeder der $n-k$ verbleibenden Knoten zu jedem der $k$ ersten Knoten höchstens einmal verbunden werden.
Falls $d_i > k$ für ein $i > k$, können wir diesen Grad nicht vollständig nutzen --- sondern nur $k$ \qq{Endpunkte} zählen.
Daher fordert das Erd\H{o}s-Gallai Theorem $\Delta \le \sum_{i=k+1}^n \min(k, d_i)$.

\subsection{Havel-Hakimi Theorem}
\begin{theorem}
    Sei $\degseq = (s, t_1, \ldots, t_s, d_{s+1}, \ldots, d_n)$ eine nicht-wachsende Gradsequenz auf $n$ Knoten, d.h.
    \begin{align}
        n > s \ge t_1 \ge \ldots t_s \ge d_{s+1} \ge \ldots \ge d_n \ge 1.
    \end{align}

    \noindent
    Dann definiere $\degseq' = \text{sort}((t_1 - 1, \ldots, t_s - 1, d_{s+1}, \ldots, d_n))$ als die sortierte Gradsequenz auf $n-1$ Knoten,
    die dadurch entsteht, dass wir den ersten (= größten) Eintrag entfernen und die verbleibenden $s$ größten Einträge um eins reduzieren.
    Dann ist $\degseq$ genau dann graphisch, wenn $\degseq'$ graphisch ist.
\end{theorem}

\begin{proof}
    Wir \aside{$\degseq' \text{ ist graphisch} \Rightarrow \degseq \text{ ist graphisch}$} zeigen zunächst die Implikation \qq{$\degseq' \text{ ist graphisch} \Rightarrow \degseq \text{ ist graphisch}$}.
    Angenommen, $\degseq'$ ist graphisch, dann existiert ein einfacher Graph~$G'$, der $\degseq'$ erfüllt.
    Dann können wir einen neuen einfachen Graphen~$G$ erzeugen, in dem wir einen neuen Knoten~$u$ hinzufügen und ihn zu den Knoten mit Graden $t_1{-}1, \ldots, t_s{-}1$ verbinden.
    Der neue Knoten~$u$ hat offensichtlich Grad~$s$ und erhöht die Grade seiner Nachbarn auf $t_1, \ldots, t_s$.
    Daher erfüllt der erzeugte Graph~$G$ die Gradsequenz~$\degseq$.
    Da jede neue Kante zu Knoten~$u$ inzident ist, können sie nicht mit alten Kanten in $G'$ kollidieren.
    Weiterhin sind die neuen Kanten zu paarweise verschiedenen existierenden Knoten inzident --- auch hier können sich weder Schleifen noch Mehrfachkanten bilden.

    Nun \aside{$\degseq \text{ ist graphisch} \Rightarrow \degseq' \text{ ist graphisch}$} zur Gegenrichtung \qq{$\degseq \text{ ist graphisch} \Rightarrow \degseq' \text{ ist graphisch}$}.
    Angenommen $\degseq = (s, t_1, \ldots, t_s, d_{s+1}, \ldots, d_n)$ ist graphisch, dann existiert ein einfacher Graph~$G$, dessen Knoten wir $S, T_1, \ldots, T_s, D_{s+1}, \ldots, D_n$ (anhand des offensichtlichen Mappings) nennen.
    Wenn $S$ zu $T_1, \ldots T_s$ verbunden ist, sind wir fast fertig: wir entfernen $S$ und alle inzidenten Kanten und erhalten den Graphen $G'$, der $\degseq'$ erfüllt.

    Im Allgemeinen ist dies aber nicht richtig, d.h. es existiert ein $T_i$ das \emph{nicht} adjazent zu $S$ ist.
    Wir konstruieren nun aus $G$ einen einfachen Graphen $G_i$, in dem die Kante $\set{S, T_i}$ existiert ohne zuvor ggf. existierenden Kanten $\set{S, T_j}$ zu verändern.

    Fixiere also ein $i$, s.d. die Kante $(S, T_i)$ nicht existiert, und ein $j$, s.d. die Kante $\set{S, D_j}$ existiert.
    Aufgrund der Sortierung von~$\degseq$ gilt $t_i \ge d_j$:
    \begin{figure}
        \begin{center}
            \begin{tikzpicture}[
                    node distance=4em,
                    node/.style={draw, inner sep=0, minimum width=1.5em, minimum height=1.5em, circle},
                    edge/.style={draw, thick, black},
                    nonedge/.style={draw, dashed, red},
                ]

                \node[node] (S0) at (0,0) {$S$};
                \node[node, above right of=S0] (T0) {$T_i$};
                \node[node, below right of=S0] (D0) {$D_j$};
                \node[node, below right of=T0] (X0) {$X$};

                \path[edge] (S0) to (D0);
                \path[edge] (T0) to (X0);
                \path[nonedge] (S0) to (T0);
                \path[nonedge] (D0) to (X0);

                %%%

                \node[node] (S1) at (15em,0) {$S$};
                \node[node, above right of=S1] (T1) {$T_i$};
                \node[node, below right of=S1] (D1) {$D_j$};
                \node[node, below right of=T1] (X1) {$X$};

                \path[nonedge] (S1) to (D1);
                \path[nonedge] (T1) to (X1);
                \path[edge] (S1) to (T1);
                \path[edge] (D1) to (X1);


                \node at ($(X0)!0.5!(S1)$) {\Large $\Rightarrow$};
            \end{tikzpicture}
        \end{center}
        \caption{
            Edge Switch im induzierten Teilgraph auf Knoten $S$, $T_i$, $D_j$ und $X$
            (rot gestrichelt zeigt an, dass hier keine Kante existiert).
            Beachte, dass sich die Grade der Knoten nicht ändern.
        }
        \label{fig:hh_edge_switch}
    \end{figure}

    \begin{itemize}
        \item
              Falls $t_i = d_j$ können wir einfach die Knoten $T_i$ und $D_j$ tauschen und sind fertig.
        \item
              Falls $t_i > d_j$ hat $T_i$ echt mehr Nachbarn als $D_j$; daher existiert ein Nachbar $X$ der mit $T_i$ aber nicht $D_j$ verbunden ist.
              Wie in \cref{fig:hh_edge_switch} gezeigt, führen wir einen sog. Edge Switch aus und ersetzen die Kanten $\set{S, D_j}$ und $\set{T_i, X}$ durch $\set{S, T_i}$ und $\set{D_j, X}$.
              Hierdurch findet keine Gradveränderung statt.
    \end{itemize}

    In beiden Fällen erhalten wir also einen Graphen in dem nun $S$ zu $T_i$ verbunden ist.
    Durch Mehrfachanwendung dieser Konstruktion erhalten wir schlussendlich einen Graphen, in  dem $S$ genau zu allen $T_j$ verbunden ist und können wie zuvor verfahren.
\end{proof}

Das Havel-Hakimi Theorem lässt sich direkt in einen Generator übersetzen.
Wir verwalten Knoten mit ihren Graden als Gewicht in einer Max-Priority-Queue.
Bis diese leer ist, führen wir die folgende Schritte aus:
\begin{itemize}
    \item Entnehme einen Knoten $S$ mit größtem Grad~$s$
    \item Entnehme $s$ schwerste Knoten $T_i$ mit Grad~$t_i$ und speichere sie als $(T_i, t_i)$ in einen Puffer
    \item Gebe für alle~$i$ die Kanten $\set{S, T_i}$ aus
    \item Für alle $i$, falls $t_i > 1$ füge Knoten $T_i$ mit Gewicht~$t_1 - 1$ erneut ein
\end{itemize}

Mit klassischen PQs (z.B. Max-Heap) lässt sich hierdurch eine Laufzeit von $\Oh{(m + n) \log n}$ erzielen;
unter der Annahme, dass die Eingabe Knoten von Grad~0 erhält vereinfacht sich der Term zu $\Oh{m \log n}$.

Da die Schlüssel in der PQ jedoch ganzzahlig sind und aus dem Intervall $[1, n)$ stammen, können wir eine Bucket-Queue verwenden.
So kann etwa $\degseq$ als doppeltverkettete Liste dargestellt werden.
Wenn $\degseq$ sortiert ist, befinden sich alle Einträge mit gleichem Grad in einer zusammenhängenden Gruppe.
Wir unterhalten dann eine zweite sortierte Liste, die jeweils auf das erste Element der Gruppe zeigt.
Durch diese Konstruktion ist es möglich, alle benötigten Operationen in Zeit~$\Oh{1}$ auszuführen --- der Generator läuft also in optimaler Laufzeit~$\Oh{m}$.

Leider sind Listen in praktischen Implementierungen aufgrund der zahlreichen Allokation und mangelnden Lokalität fast immer ein Performance-Bottleneck.
Wenn $\degseq$ sortiert ist und die Anzahl der \emph{verschiedenen} Grade $N_\degseq \ll n$ deutlich kleiner als die Anzahl der Knoten ist, ergibt sich hier ein Optimierungspotential.
Statt jeden Knoten einzeln zu speichern, verzichten wir auf die \qq{lange} lineare Liste und speichern stattdessen für jede Gradgruppe drei Informationen ab:
den Grad, die kleinste Knoten-ID in der Gruppe, die Anzahl der Knoten in der Gruppe.

Dann brauchen wir noch einen kleinen Trick: wenn wir Knoten~$T_i$ auswählen beginnen wir am Ende einer Gruppe.
Das führt dazu, dass $\degseq$ auch nach Updates monoton bleibt; außerdem kann man zeigen, dass sich während der Ausführung des Algorithmus die Länge der Liste höchstens verdoppelt~\cite{DBLP:journals/jea/HamannMPTW18}.

